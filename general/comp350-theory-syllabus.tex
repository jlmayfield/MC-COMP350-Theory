\documentclass[10pt]{article}
\usepackage{amsmath}
\usepackage{setspace}
\usepackage{hyperref}
\usepackage{booktabs}


\setlength{\textheight}{9in} \setlength{\topmargin}{-.5in}
\setlength{\textwidth}{6.5in} \setlength{\oddsidemargin}{0in}
\setlength{\evensidemargin}{0in}

\title{Syllabus \\ COMP 350 \\ The Theory of Computation}
\author{ James \textit{Logan} Mayfield }
\date{ Spring 2017 }

\begin{document}
\maketitle

\section{Logistics}
\begin{itemize}
\item \textbf{Where: } Center for Science and Business, Room 309
\item \textbf{When: } MTWF 11:00--11:50am
\item \textbf{Instructor: } James \textit{Logan} Mayfield\begin{itemize}
\item \textit{Office: } Center for Science and Business (CSB), Room 344
\item \textit{Phone: } 309-457-2200 % chktex 8
\item \textit{Website: } \url{http://jlmayfield.github.io/}
\item \textit{Email: } lmayfield \textit{at} monmouthcollege \textit{dot} edu
\item \textit{Office Hours: }  Monday and Tuesday 3--4pm. Thursday 9--10am. By Appointment.
\end{itemize}
\item \textbf{Website: } \url{http://jlmayfield.github.io/teaching/COMP350-Theory/}
\item \textbf{Credits: } 1 Course Credit
\end{itemize}
\emph{Note: This Syllabus is subject to change based on specific class needs. Deviations from the syllabus will be discussed in class.}


\section{Text}

Sipser, Michael. \textit{Introduction to the Theory of Computation}. Third Edition. Cengage Learning. 2013. ISBN-10: 1-133-18779-X % chktex 8


\section{Description and Content}

Theoretical Computer Science studies the big questions: what is computation, what can be computed, and how efficiently can it be computed.  In this course we will explore the basic tools of the computing theory with an emphasis on computability. Students that successfully complete this course will be able to:
\begin{itemize}
\item compare and contrast standard models of computation
\item prove basic computability results
\item discuss of the role of determinism and nondeterminism in computation
\item identify the importance of the halting problem
\item discuss the role of theory in Computer Science
\end{itemize}
Students will explore course topics through regular lectures, in-class problem solving sessions, and homework.  Throughout the course of the semester students will carry out independent research that expands on a topic covered in the course.  This research culminates in a paper and presentation.


\subsection{Content}

This course will cover the following:
\begin{itemize}
\item Determinism and Nondeterminism
\item Regular Languages, Finite Automata, and Regular Expressions [Chapter 1]
\item Context-Free Grammars, Context-Free Languages, and Pushdown Automata [Chapter 2]
\item Turing Machines and the Church-Turing Thesis[Chapter 3]
\item Decidability [Chapter 4]
\item Time Complexity, NP-Completeness, and P $?=$ NP [Chapter 7]
\item Space Complexity [Chapter 8]
\end{itemize}
Time permitting we'll look at selections from the remaining chapters.

\section{Expectations and Policies}

Students are expected to carry themselves in a mature and professional manner in this course. Towards this end, there are a few classroom policies by which every student is expected to abide.
\begin{itemize}

\item \textit{Late Assignments: } In general, late assignments will \textit{not} be accepted.  Students who feel they have a justified reason for submitting an assignment late may set up an appointment to meet with the instructor and plead their case.  Students are more likely to get extensions on assignments when they are asked for in advance rather than the day the assignment is due.

\item \textit{Attendance: } \textbf{Repeated absences and late arrivals to class will quickly reduce a student's participation grade to zero.}  The occasional late arrival or missed class is one thing, but being habitually late and regularly missing classes is disruptive and not fair to your classmates.

\item \textit{Participation: }  Cellphone and computer usage in class for non-class related activities is strongly discouraged.  All devices should be set to silent when in class.  If a student's usage of technology becomes a distraction to their classmates or the instructor, then that student's participation grade will suffer.  If the instructor or a classmate has to inform a student that they're being a distraction, then their use of technology has already gone too far.  When in doubt, err on the side of caution.

\item \textit{Quality of Work:} There are several minimal requirements that course assignments must meet.
\begin{itemize}

\item \textit{Staples:} Printed assignments that take up more than one page must be stapled.  Multi-page  assignments lacking staples will either be returned to the student to be stabled ASAP or points will be deducted.

\item \textit{Neatness:}  Students should make every attempt to make their work neat and orderly. When doing hand written problems or exams, label problems, avoid excessive scratching out of mistakes (use a pencil if corrections are to be expected).

\item \textit{Show Work:} Rarely are answers alone sufficient for full credit.  Show your work whenever prudent.  If you're unsure if work is needed, \textit{ask!}
\end{itemize}

\end{itemize}


\subsection{Collaboration}

In general, students are encouraged to make use of the resources available to them.  This means it is OK to seek help from a friend, the tutor, the instructor, the internet, etc.  However, \textit{copying of answers and any act worthy of the label of ``cheating'' or ``plagiarism'' is never permissible!. Students should always be able to reproduce an answer on their own, and if they cannot then they likely \textbf{do not really known the material.}} All of the Monmouth College rules on academic dishonesty apply.  A student found in violation of the rules should be prepared to face the consequences of their actions. If a student needs help understanding the rules, then please seek out the instructor before doing something that might violate academic honesty policies.

\section{Grades}

This courses uses a standard grading scale.  Assignments and final grades will not be curved except in rare cases when its deemed necessary by the instructor.  Percentage grades translate to letter grades as follows:

\begin{center}
\begin{small}
\begin{tabular}{lcl}
Score & & Grade \\ \toprule
94--100 & & A \\
90--93 & & A- \\
88--89 & & B+ \\
82--87 & & B \\
80--81 & & B- \\
78--79 & & C+ \\
72--77 & & C \\
70--71 & & C- \\
68--69 & & D+ \\
62--67 & & D \\
60--61 & & D- \\
0--59 & & F
\end{tabular}
\end{small}
\end{center}


You are always welcome to challenge a grade that you feel is unfair or calculated incorrectly.  Mistakes made in your favor will never be corrected to lower your grade.  Mistakes made not in your favor will be corrected.  \textit{Basically, after the initial grading your score can only go up as the result of a challenge.}

\subsection{Workload}

The course workload is as follows:
\begin{center}
  \begin{tabular}{lc}
    Category & Number of Assignments \\ \toprule
    Problem Sets & 6--8 \\
    Formal Proof Write-Ups & 6--8 \\
    Exams & 6--8
  \end{tabular}
\end{center}

A formal proof write-up is a typed, well worked presentation of a mathematical proof and the problem surrounding the proof. The goal is to not just develop your skill as a proof writer but develop your skills presenting and communicating proofs. A separate document will be made available that discusses the form and function of these documents.

There will be no dedicated midterm or final exam. There are just exams.  All exams will focus on material covered since the previous exam.

\subsubsection{Problem Set Grades}

Problem sets are graded on a simple three point scale where maximum credit is earned through quality work regardless of the outcome.  This means an incomplete or incorrect problem that exhibits good, well-reasoned, and well-executed work can be worth as much or more than a poorly presented correct solution. The scores are averaged and converted to standard letter grades as shown below.

\begin{center}
\begin{small}
\begin{tabular}{ll}
Assignment Avg. (Min) & Letter Grade \\ \toprule
2.8   & A  \\
2.75    & A- \\
2.5 & B+ \\
2.25    & B  \\
2   & B- \\
1.75    & C+ \\
1.5 & C  \\
1   & C- \\
0.75    & D  \\
0.5  & F
\end{tabular}
\end{small}
\end{center}


\subsection{Grade Weights}

Your final grade is based on a weighted average of particular assignment categories.  You should be able to estimate your current grade based on your scores and these weights.  You may always visit the instructor \textit{outside of class time} to discuss your current standing.

\begin{center}
  \begin{tabular}{ll}
  Category & Weight \\ \toprule
    Exams & 45\% \\ %7.5 each
    Proofs & 30\% \\ %5 each
    Problems & 20\% \\ %3+1/3 each
    Participation & 5\%
  \end{tabular}
\end{center}


\subsection{Course Engagement Expectations}

The weekly workload for this course will vary by student but on average should be about 13 hours per week.  The follow tables provides a rough estimate of the distribution of this time over different course components for a 16 week semester.
\begin{center}
\begin{tabular}{lll}
Assignment Type & Total Time & Time/week \\ \toprule
Lectures &      & 4 hours/week \\
Problem Sets & 45 hours        & 3 hours/week \\
Proofs & 45 hours        & 3 hours/week \\
Exam Study Time & 16 hours  & 1 hours/week \\
Reading+Unstructured Study & & 2 hours/week \\
\bottomrule
& & 13 hours/week
\end{tabular}
\end{center}


\subsection{Calendar}

The following calendar should provides an outline for the distribution of assignments and work throughout the semester.  \textit{This calendar is subject to change based on the circumstances of the course.}
\begin{center}
\begin{tabular}{lll}
Week & Dates & Assignments \\ \toprule
1 & 1/16--1/20 &    \\
2 & 1/23--1/27 &  P. Set 1 Due \\
3 & 1/30--2/3 &   Proof 1 Due. Exam 1. \\
4 & 2/6--2/10 &   P. Set 2 Due. \\
5 & 2/13--2/17 & Proof 2 Due. Exam 2. \\
6 & 2/20--2/24 & P Set 3 Due.  \\
7 & 2/27--3/2 & Proof 3 Due. Exam 3. \textit{No Class Friday.} \\
 & 3/6--3/10 & \textit{Spring Break} \\
8 & 3/13--3/17 &  \\
9 & 3/20--3/24 &  P. Set 4 Due. \\
10 & 3/27--3/31 &  Proof 4 Due. Exam 4.  \\
11 & 4/3--4/7 &  P. Set 5 Due.  \\
12 & 4/4--4/8 &  Proof 5 Due. \\
13 & 4/10--4/13 & Exam 5.  \textit{No Class Friday.}   \\
14 & 4/18--4/21 & P. Set 6 Due.\textit{No Class Monday.}  \\
15 & 4/24--4/28 & Proof 6 Due. \\
16 & 5/1--5/3 &   \\ \midrule
   & 5/9 & Exam 6.  \textit{(3:00--6:00pm)}  \\
\end{tabular}
\end{center}

\end{document}
