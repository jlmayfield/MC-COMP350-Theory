\documentclass[nobib]{tufte-handout}
\usepackage{amsmath}


%\setlength{\textheight}{9in} \setlength{\topmargin}{-.5in}
%\setlength{\textwidth}{6.5in} \setlength{\oddsidemargin}{0in}
%\setlength{\evensidemargin}{0in}

\title{Syllabus \\ COMP 350 --- The Theory of Computation \\ \large{Course by Arrangement} }
\author{ James Logan Mayfield }
\date{ Fall 2017 }

\begin{document}
\maketitle

\section{Logistics}
\begin{itemize}
\item \textbf{Where: } Center for Science and Business, Room 344
\item \textbf{When: } TBD
\item \textbf{Instructor: } James \textit{Logan} Mayfield\begin{itemize}
\item \textit{Office: } Center for Science and Business (CSB), Room 344
\item \textit{Phone: } 309-457-2200 % chktex 8
\item \textit{Website: } \url{http://jlmayfield.github.io/}
\item \textit{Email: } lmayfield \textit{at} monmouthcollege \textit{dot} edu
\item \textit{Office Hours: }  By Appointment.
\end{itemize}
\item \textbf{Website: } \url{http://jlmayfield.github.io/teaching/COMP350/}
\item \textbf{Credits: } 1 Course Credit
\end{itemize}
\emph{Note: This Syllabus is subject to change based on specific class needs. Deviations from the syllabus will be discussed in class.}


\section{Text}

Arora, Sanjeev \& Barak, Boaz. \textit{Computational Complexity: A Modern Approach}. Cambridge Press. 2009. ISBN-10: 978-0-521-42426-4 % chktex 8


\section{Description and Content}

Theoretical Computer Science studies the big questions: what is computation, what can be computed, and how efficiently can it be computed.  In this course we will explore some of the basic tools of computing theory with an emphasis on complexity theory, the study of efficiency. Special attention will be paid to results and ideas that shed light on probabilistic computation, quantum computation, parallel computation, and machine learning.

\subsection{Content}

This course will cover the following:
\begin{itemize}
  \item Computational Models
  \item NP \& NP-Completeness
  \item Randomized Computation
  \item Boolean Circuits
  \item Quantum Computation
  \item The PCP Theorem
  \item Computational Learning Theory
\end{itemize}
Time and interest permitting, we'll look at selections from the remaining chapters and other topics encountered in supplementary materials.

\section{Expectations and Policies}

Students are expected to carry themselves in a mature and professional manner in this course. Towards this end, there are a few classroom policies by which every student  is expected to abide.
\begin{itemize}

\item \textit{Late Assignments: } Late assignments will \textit{not} be accepted.  Students who feel they have a justified reason for submitting an assignment late may set up an appointment to meet with the instructor and plead their case.  Students are more likely to get extensions on assignments when they are asked for in advance rather than the day the assignment is due.

\item \textit{Attendance: } As this is a course by arrangement and can only proceed based on face-to-face meetings between the student and the instructor, \textit{unexcused absences will result in a penalty of 5\% from the student's final grades.}

\end{itemize}

\subsection{Workload}

The course workload is as follows:
\begin{center}
  \begin{tabular}{lc}
    \underline{Category} & \underline{Number of Assignments} \\
    Problem Sets & 8 \\
    Formal Proof Write-Ups & 6 \\
    Papers & 2
  \end{tabular}
\end{center}

A formal proof write-up is a typed, well worked presentation of a mathematical proof and the problem surrounding the proof. In short, it's a mathematical mini-paper. The goal is to not just develop your skill as a proof writer but develop your skills presenting and communicating proofs.

There are no exams. In their place you'll be doing two 6--8 page research papers: one at midterm and one at finals time.

\section{Grades}

This courses uses a standard grading scale.  Assignments and final grades will not be curved except in rare cases when its deemed necessary by the instructor.  Percentage grades translate to letter grades as follows:

\begin{center}
\begin{small}
\begin{tabular}{lcl}
\underline{Score} & & \underline{Grade} \\
94--100 & & A \\
90--93 & & A- \\
88--89 & & B+ \\
82--87 & & B \\
80--81 & & B- \\
78--79 & & C+ \\
72--77 & & C \\
70--71 & & C- \\
68--69 & & D+ \\
62--67 & & D \\
60--61 & & D- \\
0--59 & & F
\end{tabular}
\end{small}
\end{center}


You are always welcome to challenge a grade that you feel is unfair or calculated incorrectly.  Mistakes made in your favor will never be corrected to lower your grade.  Mistakes made not in your favor will be corrected.  \textit{Basically, after the initial grading your score can only go up as the result of a challenge.}

\subsection{Problem Set Grades}

Problem sets are graded on a simple three point scale where maximum credit is earned by quality work and awareness of the problems relation to the material at large and not solely on correctness.  This means an incomplete or incorrect problem that exhibits good, well-reasoned, and well-executed work can be worth as much or more than a poorly presented correct solution. A more complete rubric will be provide with the problem sets themeselves. The scores are averaged and converted to standard letter grades as shown below.

\begin{center}
\begin{small}
\begin{tabular}{cc}
\underline{Assignment Avg. (Min)} & \underline{Letter Grade} \\
2.8   & A  \\
2.75    & A- \\
2.5 & B+ \\
2.25    & B  \\
2   & B- \\
1.75    & C+ \\
1.5 & C  \\
1   & C- \\
0.75    & D  \\
0.5  & F
\end{tabular}
\end{small}
\end{center}


\subsection{Grade Weights}

Your final grade is based on a weighted average of particular assignment categories.  You should be able to estimate your current grade based on your scores and these weights.  You may always visit the instructor \textit{outside of class time} to discuss your current standing.

\begin{center}
  \begin{tabular}{ll}
  \underline{Category} & \underline{Weight} \\
    Papers & 45\% \\ %22.5 each
    Proofs & 30\% \\ %5 each
    Problems & 25\% \\ %3+1/3 each
  \end{tabular}
\end{center}


\subsection{Course Engagement Expectations}

The weekly workload for this course will vary by student but on average should be about 13 hours per week.  The follow tables provides a rough estimate of the distribution of this time over different course components for a 16 week semester.
\begin{center}
\begin{tabular}{lll}
\underline{Assignment Type} &  & \underline{Time/week} \\
Lectures \& Meetings &      & 2 hours \\
Problem Sets &    & 4 hours \\
Proofs &   & 4 hours \\
Papers &  & 1 hours \\
Reading+Unstructured Study & & 2 hours \\
\end{tabular}
\end{center}


\subsection{Calendar}

The following calendar provides an outline for the distribution of assignments and work throughout the semester.  \textit{This calendar is subject to change based on the circumstances of the course.}

\begin{center}
\begin{tabular}{llll}
\underline{Week} & \underline{Dates} & \underline{Chapters} & \underline{Assignments} \\

1 & 8/22 --- 8/25 &  Arora \& Barak Chap. 1 & PSet 1\\

2 & 8/28 --- 9/1 &  Arora \& Barak Chap. 1,2 & PSet 1\&2. Proof 1.\\

3 & 9/4 --- 9/8 &  Arora \& Barak Chap. 2 & PSet 2. Proof 2.\\

4 & 9/11 --- 9/15 & Arora \& Barak Chap. 7 & PSet 3\\

5 & 9/18 --- 9/22 &  Arora \& Barak Chap. 7 & PSet 3. Proof 3.\\

6 & 9/25 --- 9/29 &  Arora \& Barak Chap. 6 & PSet 4\\

7 & 10/2 --- 10/6 &   Arora \& Barak Chap. 6 & PSet 4. Paper 1.\\

8 & 10/9 --- 10/10 &  FALL BREAK (WThF)\\

9 & 10/16 --- 10/20 &  Arora \& Barak Chap. 10 & PSet 5.\\

10 & 10/23 --- 10/27 &  Arora \& Barak Chap. 10 & PSet 5. Proof 4.\\

11 & 10/30 --- 11/3 &  Arora \& Barak Chap. 11 & PSet 6. \\

12 & 11/6 --- 11/10 &  Computational Learning Theory & PSet 7. Proof 5\\

13 & 11/13 --- 11/17 &   Computational Learning Theory & PSet 7. \\

14 & 11/20 --- 11/21 &  THANKSGIVING BREAK (WThF) & PSet 8. Proof 6.\\

15 & 11/27 --- 12/1 &  Computational Learning Theory &  PSet 8.\\

16 & 12/4 --- 12/6 & READING DAY (Th) & \\

Final's Week & 12/12 &  & Paper 2. \\

\end{tabular}
\end{center}

\end{document}
