\documentclass{tufte-handout}
\usepackage{amsmath}
\usepackage{hyperref}
\usepackage{booktabs}

\title{COMP 350 --- Theory \\  Proof Papers}
\author{  }
\date{Spring 2017}

\begin{document}
\maketitle

\begin{abstract}
One of the major goals of this course is develop your skills as a writer and presenter of mathematical proofs. Towards this end you will do detailed write-ups of several proofs throughout the semester. Think of these as mini-mathematical papers. This document provides the framework for these papers and discusses how they will be evaluated.
\end{abstract}

\section{Paper Structure}

We'll use a prescribed structure for these papers so that you may focus on the content more so than the highest-level details of the paper. Each paper will have four sections: the presentation of the problem, the sketch of the proof, the proof itself, and concluding remarks. All but the proof section should be no more than a paragraph or two. The content of these sections is discussed in the remainder of the section. The next section discusses writing of the proof itself.

\subsection{ Problem Presentation }

Before we get into the the nitty-gritty details of the proof we should take a moment to present the problem to the reader an offer some context. Our goal is to make clear what we're going to do and why. This presentation should include an informal statement of the problem, examples to illustrate the problem, a brief statement of the relevance of the problem, and finally a formal statement of the theorem we're trying to prove.  In a nutshell, this section should quickly run through the problem informally and then develop a formal statement of the problem as a mathematical theorem.

\subsection{ Proof Sketch }

Where the problem statement makes clear what we're going to prove, the proof sketch must make clear how we're going to prove it. Our text book is generally very good at illustrating and utilizing an informal sketch of the proof prior to the formal proof so you need go no further than it for examples.

The proof sketch should provide the reader with a complete picture of the overall ebb and flow of the proof to come and highlight the key parts of its argument. You absolutely must describe the style of proof and and key techniques used.  When done well, a proof sketch acts as a guide to reading the proof. It tells the reader what kind of proof they're going to read and how key elements of the problem map to that proof structure. If you'd like a focused look at basic proof techniques and styles, then I highly recommend Hammack's excellent \textit{free} text on the subject\cite{Hammack2013}.

\subsection{ Concluding Remarks }

The concluding remarks serve two purposes. They summarize the work done and then connect it to potential future work. In summarizing the theorem and its proof we open the door for connecting the content of the theorem to new ideas or the techniques of the proof to other theorems. The important thing is that we take a moment at the end, when all should be crystal clear, to put the work in context and don't simply present that work in a vacuum. When done right, the problem presentation and the concluding remarks work together to place the theorem and proof into big picture.

\section{Structured Proofs}

We'll be using a style of proof writing advocated by Leslie Lamport\cite{Lamport2012,Lamport1993} and Uri Leron\cite{Leron1983}. The basic idea is to design and present your proof in a structured, \textit{hierarchical} fashion\sidenote{much like we do with our programs} and not in a strictly linear fashion. Before you write your first proof paper, take a moment to at least scan through the Lamport and Leron papers to get a sense for the style.

Structured proofs incrementally dig down through the logic of the proof's argument from top to bottom where each layer provides a deeper look at some or all of its parent layer. Deeper layers of the proof often contain proofs of pieces of their parents. Readers looking for a formal, yet still high-level, look at the proof can read only the top layer or two.  Readers looking to rigorously verify the correctness of the proof can read the whole thing in a top-down or bottom-up fashion. In either case, the proof is not necessarily read in the same order that it's presented on the page.

The proof in your paper must clearly exhibit a hierarchical structure\sidenote{In the event that the proof requires only one layer, then you should point that out and address why this is the case for this proof.}  Lamport recommends a numbered list structure, like an outline, but you can also use subsections of increasing depth in the proof section of the paper to effectively nest the proof structure. In either case, the structure of the proof must be made clear and explicit.

The proof section of your paper \textit{must} begin with a brief paragraph describing the structure of your proof, i.e.\ depth of the hierarchy and role of each level.  The previously given proof sketch informally lays down the high-level logic of the argument of the proof where this paragraph maps that argument to a hierarchical structure.

The ground floor of the proof in your paper is the proven theorems and definitions of the text or explicitly cited work in theoretical computer science and mathematics. In the class we'll try to strike a balance between deep rigor as described by Lamport and brevity and high-level clarity.

\section{Logistics}

All proof papers must be typed and double-spaced. Only printed hard-copies will be accepted. When used, diagrams and equations should be computer generated and not hand written. This means you'll need to figure out an effective way to generate equations and diagrams for your papers.  If you want to use the tools used by professionals, then now is the time to learn \LaTeX. If you don't wish to take that path then everything you need to do can be accomplished within the suite of Microsoft office tools. The total length of the paper will vary depending on the size and amount of figures and diagrams used, but 2--4 pages is a probably a pretty good guide\sidenote{Note that all but the proof section should be limited to a paragraph or two}.

Each paper is graded on a standard 100 point scale with the following rough rubric that emphasizes the deeper presentation of the proof over the ultimate outcome.

\begin{center}
  \begin{tabular}{ll}
    \underline{Category} & \underline{Weight} \\
      Problem Presentation & 10\% \\
      Proof Sketch & 30\% \\
      Presentation of Proof Structure & 10\% \\
      Quality and Clarity & 20\% \\
      of Proof Structure \\
      Correctness of Proof & 20\% \\
      Conclusions & 10\%
  \end{tabular}
\end{center}

%\subsection{\LaTeX}

%\url{https://www.tug.org/texlive/}
%\url{http://www.xm1math.net/texmaker/}
%\url{http://www.math.harvard.edu/texman/}
%\url{http://www.texample.net/tikz/examples/state-machine/}
%\url{http://mirror.hmc.edu/ctan/macros/latex/required/amscls/doc/amsthdoc.pdf}


\bibliographystyle{plain}
\bibliography{comp350.bib}
\end{document}
