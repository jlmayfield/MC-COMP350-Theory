\documentclass[nobib]{tufte-handout}
\usepackage{amsmath,amssymb,amsthm}


\title{COMP 350 --- Theory of Computation \\ Problem Set 02}

\begin{document}
\maketitle

\begin{center}
\begin{tabular}{l}
  \underline{Exercises \& Problems} \\
  2.1 \\
  2.2* \\
  2.7 (Proof Problem)\\
  2.9 \\
  2.10 \\
  2.17* \\
  2.21* \\
  2.23 \\
  2.24 \\
  2.25 \\
  2.27 \\
  2.33*
\end{tabular}
\end{center}

Exercies marked with a * are about specific laguages. Before working to solve the exercies as stated, develop a set of concrete instances of the language with an eye towards illustrating the crux of the exercise. 

\section*{Problem Set Rubric}

The problem set is worth three points.  One and a half point comes from your meta-cognitive analysis of the problem set, three quarters of a point comes from attempting a sufficient sample of the problems, and the three quarters of a point comes from the correctness and quality of the problem.

\subsection*{Analysis}

The goal of your meta-cognitive analysis is to look at the problems assigned, evaluate their purpose, and place them in the wider context of the material from the chapter. Minimally, you should: connect each problem to theorems, definitions, and examples from the chapter, attribute the problem to a more general problem/task in theory discussed in the context of the chapter, and weigh the problem's relative importance within the set. A good analysis should connect the specific problem to more general ideas and techniques presented in the chapter and the course. Finally, you analysis should identify a minimal sample of problems from the set that are critical to understanding the material and justify each problem's inclusion in this sample.

Use no more than a page (\textbf{typed}) to present your analysis. Alternatively, you can create a concept map or some other visualization to capture the results of your analysis.

\subsection*{Problems Attempted}

In the course of your meta-cognitive analysis you should identify a core set of problems from the problem set.  To get full credit for attempting problems you must attempt these problems and these problems should, in fact, be a good representative sample of problems for the chapter. Alternatively, you can take the shotgun approach and go above and beyond this set of problems.

\subsection*{Correctness and Quality}

Full credit for the problem set requires that you meet some level of success on the problems you attempt and that the quality of the work you do be relatively high. Success can mean doing the problem correctly or failing to do the problem correctly but displaying some understanding of where things go wrong.  This later form of success is likely to require a bit of meta-cognitive analysis and a sentence of two discussing where and why you got stuck. High Quality work should be neat, organized and make good use of both prose and mathematical formalism.

\textit{Given the individualized nature of this course, the evaulation of individual problems will take place at in-office meeting times. You should be prepared to discuss and present problems on the problem set due dates.}



\end{document}
